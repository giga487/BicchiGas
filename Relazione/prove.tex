\documentclass[a4paper]{article}
\usepackage[T1]{fontenc}
\usepackage[utf8]{inputenc}
\usepackage[english,italian]{babel}
\usepackage{amsmath}
\usepackage{amssymb}
\usepackage{array}
\usepackage{booktabs}
\usepackage{mathtools}
\usepackage{diffcoeff}
\usepackage{caption}
\usepackage{float}
\usepackage{titlesec}
\usepackage{gensymb}
\usepackage{siunitx}
\usepackage{subfig}

\begin{document}

\begin{equation}
\tau = K_p e + K_d \dot{e} + G(q)
\end{equation}
\begin{equation}
K_p = 
\begin{bmatrix}
1000 & 0 & 0	& 0 & 0 & 0 \\
0 & 1000 & 0 & 0 & 0 & 0 \\
0 & 0 & 100 & 0 & 0 & 0 \\
0 & 0 & 0 & 100 & 0 & 0 \\
0 & 0 & 0 & 0 & 100 & 0 \\
0 & 0 & 0 & 0 & 0 & 10
\end{bmatrix}
\end{equation}
\begin{equation}
K_d = 
\begin{bmatrix}
10 & 0 & 0	& 0 & 0 & 0 \\
0 & 100 & 0 & 0 & 0 & 0 \\
0 & 0 & 10 & 0 & 0 & 0 \\
0 & 0 & 0 & 10 & 0 & 0 \\
0 & 0 & 0 & 0 & 10 & 0 \\
0 & 0 & 0 & 0 & 0 & 10
\end{bmatrix}
\end{equation}

\begin{equation}
\tau = B(q)[\ddot{q_d} + K_p e + K_d \dot{e}] + C(q,\dot{q})\dot{q} + G(q)
\end{equation}
\begin{equation}
K_p = 
\begin{bmatrix}
1 & 0 & 0	& 0 & 0 & 0 \\
0 & 1 & 0 & 0 & 0 & 0 \\
0 & 0 & 1 & 0 & 0 & 0 \\
0 & 0 & 0 & 1 & 0 & 0 \\
0 & 0 & 0 & 0 & 1 & 0 \\
0 & 0 & 0 & 0 & 0 & 1
\end{bmatrix}
\end{equation}
\begin{equation}
K_d = 
\begin{bmatrix}
1 & 0 & 0	& 0 & 0 & 0 \\
0 & 1 & 0 & 0 & 0 & 0 \\
0 & 0 & 1 & 0 & 0 & 0 \\
0 & 0 & 0 & 1 & 0 & 0 \\
0 & 0 & 0 & 0 & 1 & 0 \\
0 & 0 & 0 & 0 & 0 & 1
\end{bmatrix}
\end{equation}

\begin{equation}
\tau = Y \hat{\pi} + K_d\dot{e} + K_p e
\end{equation}
\begin{equation}
u_{\pi} = R^{-1} Y^T M^{-T} B^T P x
\end{equation}
\begin{equation}
K_p = 
\begin{bmatrix}
1000 & 00 & 0	& 0 & 0 & 0 \\
0 & 10000 & 0 & 0 & 0 & 0 \\
0 & 0 & 1000 & 0 & 0 & 0 \\
0 & 0 & 0 & 1000 & 0 & 0 \\
0 & 0 & 0 & 0 & 1000 & 0 \\
0 & 0 & 0 & 0 & 0 & 10
\end{bmatrix}
\end{equation}
\begin{equation}
K_d = 
\begin{bmatrix}
100 & 00 & 0	& 0 & 0 & 0 \\
0 & 1000 & 0 & 0 & 0 & 0 \\
0 & 0 & 1000 & 0 & 0 & 0 \\
0 & 0 & 0 & 100& 0 & 0 \\
0 & 0 & 0 & 0 & 100 & 0 \\
0 & 0 & 0 & 0 & 0 & 10
\end{bmatrix}
\end{equation}
\begin{equation}
R = I_{6}
\end{equation}
\begin{equation}
A = 
\begin{bmatrix}
0_{6} & I_{6} \\
-K_p & -K_d 
\end{bmatrix}
\end{equation}
\begin{equation}
B = 
\begin{bmatrix}
0_{6} \\
I_{6}
\end{bmatrix}
\end{equation}
\begin{equation}
Q = I_{12}
\end{equation}
\begin{equation}
A^T P + P A = -Q
\end{equation}

\end{document}

